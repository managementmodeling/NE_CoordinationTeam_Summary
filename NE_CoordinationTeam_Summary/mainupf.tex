\documentclass[12pt,b5paper]{NE_meetings}
% two-side already included in class book, it rises an error combining caption and fltpage packages
% CODIFICACI�

%\DeclareUnicodeCharacter{FFFD}{}


%\usepackage[latin1]{inputenc}
\usepackage[T1]{fontenc}
\usepackage[utf8]{inputenc}
\DeclareUnicodeCharacter{FFFD}{}



% IDIOMES
\usepackage[catalan,italian,spanish,english]{babel}%\usepackage[english]{babel}
%\usepackage[authoryear,round,longnamesfirst]{natbib}
% citep (parenthesis)
% citet (inside text)
%\usepackage{bibentry}
%\usepackage[backend=biber]{biblatex}

%%%% BIB M
\usepackage[backend=bibtex,style=nature]{biblatex}
\renewbibmacro{in:}{}
\DeclareLanguageMapping{english}{english-apa}
\selectlanguage{english}
%

\bibliography{Paper_Ref}


\usepackage[a4paper,  inner=3.5cm,outer=3cm,top=3.5cm,bottom=3.5cm]{geometry}

%Cris   inner=3.25cm,outer=2.25cm,top=2.75cm,bottom=2.5cm
%Giulia  inner=3.5cm,outer=3cm,top=3.5cm,bottom=3.5cm


% PER A INCLOURE GR�FICS I EL LOGO DE LA UPF
\usepackage{graphicx}
\usepackage[svgnames]{xcolor}

% Extra packages.
\usepackage{amsmath}
\usepackage{dcolumn} % Align table columns on decimal point
\usepackage{bm} % bold math
\usepackage{amssymb, wasysym, dsfont}
\usepackage{textcomp}
\usepackage{pifont}
\usepackage{booktabs}
\usepackage{array}
\usepackage{multirow}
\usepackage{changepage}
\usepackage{pdfpages}
\usepackage{graphics}
\usepackage{caption}
\usepackage{eurosym} % para el euro
\usepackage{rotating}
\usepackage{setspace}
\usepackage{ulem}
\newenvironment{subsubsub} %%indent subsubsubsections
  {\adjustwidth{3em}{0pt}}
  {\endadjustwidth}

\makeatletter

\newcommand{\armultirow}[3]{%
  \multicolumn{#1}{#2}{%
    \begin{picture}(0,0)%
      \put(0,0){%
        \begin{tabular}[t]{@{}#2@{}}%
          #3%
        \end{tabular}%
      }%
    \end{picture}%
  }%
}%

\newcolumntype{f}{>{$}l<{$}}
\newcolumntype{n}{l}
\newcolumntype{N}{>{\scriptsize}l}
\newcolumntype{w}[1]{>{\raggedleft\hspace{0pt}}p{#1}}
\newcolumntype{v}[1]{>{\raggedright\hspace{0pt}}p{#1}}
\newcolumntype{V}[1]{>{\sffamily\bfseries\raggedright\hspace{0pt}}p{#1}}
%
% array.sty, dcolumn.sty
\newcolumntype{B}[1]{>{\boldmath\DC@{.}{,}{#1}}l<{\DC@end}}
\newcolumntype{d}[1]{>{\DC@{.}{,}{#1}}l<{\DC@end}}
\newcolumntype{i}[1]{>{\DC@{.}{,}{#1}\mathnormal\bgroup}l<{\egroup\DC@end}}
\newcolumntype{s}[1]{>{\DC@{.}{,}{#1}\mathsf\bgroup}l<{\egroup\DC@end}}
\makeatother

% \usepackage{times}
\usepackage[nooneline]{subfigure}
\usepackage{float}
\usepackage{rotating}
\usepackage[font=small,labelfont=bf]{caption}
% caption in a separate page
\usepackage[leftFloats, CaptionAfterwards]{fltpage}
% Wrapping figures
\usepackage{wrapfig}
% Handling clearpages and floats (to manage proper location of floats)
\usepackage{afterpage}
% Environments with multiple args
\usepackage{xparse}

\usepackage{microtype}          % To avoid undesired line breaks.
                                % It does not solve the problem
                                % completely in cites
\usepackage{breakcites}         % This other package does the trick

\def\phrase{\begin{otherlanguage*}{nohyphenation}%
  \dononbreakablespace
  \dononbreakablehyphen
  \dophrase}
\def\dophrase#1{#1\end{otherlanguage*}}

{\catcode`\ =\active
\gdef\dononbreakablespace{\catcode`\ =\active\def {\nobreakspace}}}
{\catcode`\-=\active
\gdef\dononbreakablehyphen{\catcode`-=\active\def-{\nobreakhyphen}}}
\def\nobreakhyphen{\hbox{-}\nobreak}

% TikZ
\usepackage{tikz}

\usetikzlibrary{arrows,mindmap,trees,shapes,backgrounds, calc,calendar,patterns, shadings,
  shadows,shapes.geometric,decorations.markings,arrows.meta}

\usepackage{pgfplots}
\usepgfplotslibrary{groupplots,polar,fillbetween}
% \pgfplotsset{compat=1.5.1}
\pgfplotsset{compat=1.11}

% Framed parts
%\usepackage[framemethod=TikZ]{mdframed}x% Framed examples
%\mdfdefinestyle{commentstyle}{
%hidealllines=true,
%backgroundcolor=gray!20,innertopmargin=\topskip,splittopskip=\topskip,
%}
%
%\mdtheorem[hidealllines=true,
%backgroundcolor=gray!20,innertopmargin=\topskip,splittopskip=\topskip,
%]{comment}{Comment}[section]



%\input{mdframe_equation}%

% FONTS TIMES O GARAMOND, 
\usepackage{times}
% \usepackage{garamond}
% SENSE HEADINGS: NO MODIFICAR
\pagestyle{plain}

% PER A L'�NDEX DE MATURES
\usepackage{makeidx}
\makeindex

% ESTIL DE BIBLIOGRAFIA
%\bibliographystyle{unsrt}
%\renewcommand{\bibname}{biblio}

% This document is mostly in english
\selectlanguage{english}

% EN COMPTES DE �NDEX, LA TAULA DE CONTINGUTS ES TITULA SUMARI
\addto\captionscatalan
{\renewcommand{\contentsname}{\Large \sffamily Sumari}}


\usepackage{imakeidx} % subject index and index of authors
\makeindex
\makeindex[title=SUBJECT INDEX,columns=2]
%\makeindex[,title=INDEX OF AUTHORS,columns=2]
\makeindex[name=authors,title=INDEX OF AUTHORS,columns=2]

\usepackage{titlesec}
\usepackage[dotinlabels]{titletoc}
\usepackage{appendix}
\usepackage{chngcntr}
\usepackage{etoolbox}
\usepackage{lipsum}

\makeatletter
\titlecontents{chapter}[5pc]
{\addvspace{1pc}\bfseries \sffamily
  \filright}
{\contentslabel
  [\@chapapp\
  \thecontentslabel]{6pc}\hspace{-0.15cm}}
{}{\hfill\contentspage}
[\addvspace{2pt}]
\makeatother
% Show only chapter/sectionentries:
\setcounter{tocdepth}{3}
\setcounter{secnumdepth}{3}
\contentsfinish

\AtBeginEnvironment{subappendices}{%
\section*{\bf \sffamily Chapter Appendices}
\addcontentsline{toc}{section}{\hspace*{-0.7cm}\bfseries \sffamily Chapter Appendices\vspace*{0.4em}}
\counterwithin{figure}{section}
\counterwithin{table}{section}
\counterwithin{equation}{section}
}


% Different Chapter sytle
% Styles: Sonny, Lenny, Glenn, Conny, Rejne, Bjarne
\usepackage[Sonny]{fncychap}
\usepackage{epigraph}           % For introducing epigraphs
\setlength\epigraphwidth{11cm}
\setlength\epigraphrule{0pt}

\makeatletter
\g@addto@macro\appendices{%
  \renewcommand\chaptername{\sffamily Appendix}%
  \addtocontents{toc}{\protect\renewcommand{\protect\@chapapp}{\appendixname}}%
  \counterwithin{equation}{section}
}
\makeatother

\addto\captionsenglish
{\renewcommand{\contentsname}{\Large \sffamily \bfseries Contents}}
\addto\captionsenglish
{\renewcommand{\listfigurename}{\Large \sffamily \bfseries List of Figures}}


%AFEGIU EN AQUESTA PART LES VOSTRES DADES
\title{Cristina González \newline Ricardo Salvador \newline Roser Sánchez-Todo  \newline Giulio Ruffini}
\subtitle{Project coordination team}
\author{Coordination meetings summary}
\thyear{2021}


% Enumerate package
%\usepackage{enumerate}
\usepackage[inline]{enumitem}

% Custom commands
\newcommand{\mal}[1]{{\color{red}\underline{#1}}}
\newcommand{\sobra}[1]{{\color{gray}#1}}
\newcommand{\duda}[1]{{\color{blue}\textit{#1}}}
\newcommand{\cita}[1]{{\color{red}[cita requerida]}}

\usepackage{varioref}


% Hyperrefs
\usepackage[colorlinks=true, citecolor=black,
linkcolor=black,linktoc=all,plainpages=true]{hyperref}

\usepackage{cleveref}
\crefname{figure}{Fig.}{Figs.}
\crefname{equation}{Eq.}{Eqs.}
\crefname{table}{Tab.}{Tabs.}

% %%%%%%%%%%%%%%%%%%%%%%%%%%%%%%%%%%%%%%%%%%%%%%%%%%%%%%%%%%%%%%%%%%%%%%%%%%%%%%%%
\begin{document}

\pdfstringdefDisableCommands{%
\let\MakeUppercase\relax
}

\pagestyle{empty}

\maketitle
\frontmatter
\pagestyle{plain}


%%%%TAULA DE CONTINGUTS: OBLIGAT�RIA

%\addcontentsline{toc}{chapter}{Contents}
%\cleardoublepage
\setcounter{tocdepth}{1}
\tableofcontents


%%%%%% INDEX DE FIGURES; NOM�S ES POSA SI HI HA FIGURES

%\cleardoublepage
%%%%%Fa que aparegui al sumari
%%%-------------------------------------------------------------------------
%\addcontentsline{toc}{chapter}{List of figures}

%\listoffigures

%\cleardoublepage
%\addcontentsline{toc}{chapter}{List of tables}
%\listoftables
%%%-------------------------------------------------------------------------
\clearpage
\mainmatter


\chapter{Regular Meetings}
\spacing{1.1}

 Here you can find a Summary of all the coordination official meetings ordered by date:

\section{2021, January Thu.7^{th}}

Meeting: CG + RST

\subsection{NeuroTwin}
\begin{itemize}
    \item Document Rosa spots.
    \item \href{https://www.dropbox.com/home/998%20-%20neurotwin.eu%20repository/05%20-%20Deliverables?preview=NEUROTWIN+Deliverable+-+D5.1+Quality+Guide_v1.6.docx}{Deliverable D5.1} Feedback?

    \item Theoretical
        \begin{itemize}
            \item Starsim System: HW?
         	\itemStimweaver algorithm: SW? Servicio Extra a parte de Starsim. No va necesariamente unido al Starsim. Patient specific/group Specific/ Generic.
            \item	Ising (Niko) / laNMM  (MT Roser).
            \item	Concepts criticality and normalization.
            \item	Paper Pallop 2016.
        \end{itemize}
\end{itemize}


\subsection{Galvani}
\begin{itemize}
    \item How? When? 12t task?
    \item \href{https://docs.google.com/document/d/1pW-zTeiSJdutUG7ir9rBoyD5eRubBpaRs16GNvNYou4/edit?ts=5ff6cca4   }{Galvani Lab tasks/ Project logbook}: Edit document and create a template. Modify Drive document and Remind to the team members to fill it!!!! 

\end{itemize}

\subsection{Logistics \& other stuff}
\begin{itemize}
    \item Position Laura López Galdo.
    \item Revise list of PM tasks.
    \item Role in Meetings: Keep track of discussions in meetings and save it to meetings folder.

\end{itemize}

\subsection{TO DO:}
\begin{itemize}
   \item Integrate Gantt\footnote{https://prezi.com/view/A4DTYXXpVsNqS1TuUwdb/}: pattern/colors/text in cells.
    \item \href{https://www.dropbox.com/home/998%20-%20neurotwin.eu%20repository/05%20-%20Deliverables?preview=NEUROTWIN+Deliverable+-+D5.1+Quality+Guide_v1.6.docx}{Deliverable D5.1} Feedback.
    \item Logbook Galvani-put it to Slack.
    \item Better solution to lunch.
    \item Technical notes (NT): Keep track of technical notes, which are under wirtting process, missing and future. Ex.:TN0145.
    \begin{itemize}
        \item Create an excel with names of people, TN, and status.
        \item Name convention and folders.
    \end{itemize}

\end{itemize}

\subsection{DONE:}

\newpage
\section{2021, January Mo.10^{th}}


Meeting: CG + RST + RS + GR 


\subsection{NeuroTwin}
\begin{itemize}
\item \textcolor{red}{Due date: 2/3/4… etc. How do you fix an exact day of delivery? Ex.: D5.1: 11th Feb 2021 (M2).
}
\item	\textcolor{red}{BIDMC is asking for assurances about compliance with GPDR to receive non-US data at BIDMC we could get a general statement signed across institutions ensuring everyone will comply with regulations when transferring data outside EU?}
\item \textcolor{red}{Consortium Agreement?} Deadline: End Feb.
\item \textcolor{red}{Data management plan?} D3.1. Deadline: Due month 3.
\item D5.1. \textbf{Questions}

"TEMPLATES: The following templates shall be provided to the consortium:\\
1.	Presentation template (.ppt and .key)\\
2.	Deliverable template (.docx)"\\
3.  This is the Quality Control Review Form (.docx)\\

$@$Giulio: Editable tables of the grant agreement? Folder Grant Agreement


\item Jordi- Suggestions before meeting him. General idea of what we want.Create Open Project or similar system for tasks (PRT system)[Coordinate with Jordi/IT] 



\end{itemize}


\subsection{Galvani}
\begin{itemize}
    \item Meeting Tu. $12^{th}$
    \item Periodicity?: Meetings to discuss strategy and matters at hand. (3months?¿?)
    \item Galvani Steering Committee Creation /Consortioum Agreement: \textbf{Rennes} (involving Tech Transfer agencies in Rennes and Marseille, Corporate Development in NE)
  

\end{itemize}

\subsection{Logistics \&  other stuff}
\begin{itemize}
    \item \textcolor{red}{STORAGE.} \href{https://docs.google.com/document/d/16v9LeHPt3SFdeu1WQeSEbZKQP_1lh1sQ4h05umlxRZ4/edit}{Check here}
    \item \textcolor{red}{Rules for joined document revisions, control of the revisions, Filenames conventions.}\href{https://docs.google.com/document/d/1TGVl9pppeMySwDjRnmjSmvmIVzkaIl4oBJjY9_VCWMU/edit} {Check here}
     \item Is there any confidential information about Galvani/ Neurotwin that cannot be shared w/ the group?

   
\end{itemize}

\subsection{TO DO:}
\begin{itemize}
   \item Integrate Gantt: pattern/colors/text in cells.
    \item \textbf{Feedback in D5.1}.
    \item Logbook Galvani-put it to Slack \textcolor{red}{prospective or retrospective?}.
    \item Technical notes (NT): Keep track of technical notes, which are under wirtting process, missing and future. Ex.:TN0145.
    \begin{itemize}
        \item Create an excel with names of people, TN, and status.
        \item Name convention and folders.
    \end{itemize}
    \item Send an email in a month from now to discuss literature (ask for literature to Emiliano, Giacomo...)
    \item Give feedback about mice database to Javier \footnote{https://imaging.org.au/AMBMC/Model}.
    \item Organize pathophysiology meeting (in a month from now 08.01.2021 $\rightarrow$ beginning of Feb. Tentative day: Fri. 5$^{th}$)
    \item Neurotwin: make a compact calendar with deliverables (D) and milestones (MS).
    \item \textbf{Skat - Cerverus}    
    \item High level gantt: Deadlines to think about.
    \begin{itemize}
        \item 1st Week February: Project Management System.
        \item Last week February: Implemented Project Management System.
        \item Coordinate w/ Jordi and David.
    \end{itemize}
    
\end{itemize}

\subsection{DONE:}
\begin{itemize}
    \item Better solution to lunch.
    \item Rules for joined document revisions, control of the revisions, Filenames conventions.
    \item Last version: \href{https://www.dropbox.com/home/998%20-%20neurotwin.eu%20repository/05%20-%20Deliverables?preview=NEUROTWIN+Deliverable+-+D5.1+Quality+Guide_v1.7_CG.docx}{Deliverable D5.1}.
\end{itemize}
\newpage
%This is a template for completing this Summary Guide.
%Make sure you always start a new file and you do not overwrite this template.

\section{2021, January Mo.18^{th}}


Meeting: People involved in the meeting.

\subsection{NeuroTwin}
\begin{itemize}
    \item aaaaaa
    \item aaaaa

\end{itemize}


\subsection{Galvani}
\begin{itemize}
    \item aaaaaa
    \item aaaaa

\end{itemize}

\subsection{Logistics \&  other stuff}
\begin{itemize}
    \item aaaaaa
    \item aaaaa

\end{itemize}


\subsection{TO DO:}
\begin{itemize}
    \item Logbook Galvani-put it to Slack \textcolor{red}{prospective or retrospective?}.
    \item Technical notes (NT): Keep track of technical notes, which are under wirtting process, missing and future. Ex.:TN0145.
    \begin{itemize}
        \item Create an excel with names of people, TN, and status.
        \item Name convention and folders.
    \end{itemize}
    \item Send an email in a month from now to discuss literature (ask for literature to Emiliano, Giacomo...) (in a month from now 08.01.2021 $\rightarrow$ beginning of Feb).
    \item Give feedback about mice database to Javier \footnote{https://imaging.org.au/AMBMC/Model}.
    \item Organize pathophysiology meeting (in a month from now 08.01.2021 $\rightarrow$ beginning of Feb. Tentative day: Fri. 5$^{th}$)
    \item Neurotwin: make a compact calendar with deliverables (D) and milestones (MS).
    \item \textbf{Skat - Cerverus}    
    \item High level gantt: Deadlines to think about.
    \begin{itemize}
        \item 1st Week February: Project Management System.
        \item Last week February: Implemented Project Management System.
        \item Coordinate w/ Jordi and David.
        
\item \textcolor{red}{Consortium Agreement?} Deadline: End Feb.
\item \textcolor{red}{Data management plan?} D3.1. Deadline: Due month 3.
\item D5.1. \textbf{Templates}:
\begin{itemize}
    \item 	Presentation template (.ppt and .key)
    \item Deliverable template (.docx)
    \item \textcolor{red}{Quality Control Review Form (.docx)????}
\end{itemize}

\end{itemize}

\end{itemize}

\subsection{DONE:}
\begin{itemize}
    \item Integrated Gannt: LINK
    \item Last version \href{https://www.dropbox.com/home/998%20-%20neurotwin.eu%20repository/05%20-%20Deliverables?preview=NEUROTWIN+Deliverable+-+D5.1+Quality+Guide_v1.7_CG.docx}{Deliverable D5.1}.
\end{itemize}
%\newpage
%%This is a template for completing this Summary Guide.
%Make sure you always start a new file and you do not overwrite this template.

\section{YYYY, Month Day.nr^{th}}
Hola este es el segundo meeting:

Meeting: People involved in the meeting.

\subsection{NeuroTwin}
\begin{itemize}
    \item aaaaaa
    \item aaaaa
    \item aaaaa
    \item aaaaaa
    \item aaaaaaaa
\end{itemize}


\subsection{Galvani}
\begin{itemize}
    \item aaaaaa
    \item aaaaa
    \item aaaaa
    \item aaaaaa
    \item aaaaaaaa
\end{itemize}

\subsection{Logistics \&  other stuff}
\begin{itemize}
    \item aaaaaa
    \item aaaaa
    \item aaaaa
    \item aaaaaa
    \item aaaaaaaa
\end{itemize}

\subsection{TO DO:}
\begin{itemize}
    \item aaaaaa
    \item aaaaa
    \item aaaaa
    \item aaaaaa
    \item aaaaaaaa
\end{itemize}

\subsection{DONE:}
\begin{itemize}
    \item aaaaaa
    \item aaaaa
    \item aaaaa
    \item aaaaaa
    \item aaaaaaaa
\end{itemize}



\chapter{Tasks for Brain Modeling Project Coordinator}
\spacing{1.1}

 Here you can find an ongoing list for the Brain Modeling Project Coordinator:\\
 \newline
 Last modified: 2020, January 7^{th}\\
 
 \begin{itemize}
    \item	Ensure Brain Modeling projects and operations success, with objectives achieved with quality, in a timely manner and within resources. 
\item Create Open Project or similar system for tasks (PRT system)\textcolor{red}{[Coordinate with Jordi/IT]} 
\item Maintain the Project test reports (PTR) content: Project test plan reports. 
\item Organize short/executive weekly meetings with Giulio, Ricardo and Roser to review this. 
\item Keep track of \textcolor{red}{high level management milestones} - Manage/maintain Grants and Projects sheets. 
\item In Neurotwin, we are coordinators. Need to study Neurotwin to create full gantt to Task level, then support Giulio in keeping track of deliverables and milestone and reporting for Commission [coordinate with David and Roser, too] 
\item Maintain Don’t Panic Guide. 
\item Ensure TNs and project folders are properly created and executed and saved. 
\item Support Giulio with HR processes (hiring, evaluations) 
\item Last but not least, come up with new ideas to fulfill the mission!
\item Project logbook for Galvani.

\end{itemize}
\newpage

\chapter{Documents}
\spacing{1.1}


 Here you can find Templates and  Documents relevant for the project:

\section{Templates}
\begin{itemize}
    \item \href{https://docs.google.com/document/d/1pW-zTeiSJdutUG7ir9rBoyD5eRubBpaRs16GNvNYou4/edit}{Galvani logbook Template}
    \item \href{https://docs.google.com/document/d/1yougMXQcd1M_uPpjK7gK8gJw0-Tz47TKAVfftTt90z4/edit}{Thursday Weekly Group Lunch}
   
\end{itemize}
\newpage
\section{Other documents}
\subsection{Projects}
\subsubsection{Neurotwin}
\begin{itemize}
 \item \href{https://www.dropbox.com/home/998%20-%20neurotwin.eu%20repository/01%20-%20Neurotwin%20Starter%20Kit?preview=Grant+Agreement-101017716-Neurotwin.pdf}{NeuroTwin Grant Agreement.}
 \item \href{https://www.dropbox.com/home/998%20-%20neurotwin.eu%20repository/01%20-%20Neurotwin%20Starter%20Kit?preview=Grant+Agreement-101017716-Neurotwin+TECH.pdf}{NeuroTwin Grant Agreement TECH (short).}
 \item \href{https://www.dropbox.com/home/009%20-%20Neuroelectrics%20Research%20WORK%20AREA/002%20-%20Neurotwin/101%20-%20Neurotwin%20Data%20Management%20Plan/from%20Luminous}{NeuroTwin Data Management Plan.}
 \item \href{https://www.dropbox.com/home/009%20-%20Neuroelectrics%20Research%20WORK%20AREA/002%20-%20Neurotwin/102%20-%20Neurotwin%20Consortium%20Agreement}{NeuroTwin Consortium Agreement.}
 \item \href{https://docs.google.com/document/d/1lEjSbhkZhggJMzSBceajwg3VIAWfD3EKFGuKd5JcIPM/edit}{Neurotwin Web Content.}
 \item \href{https://prezi.com/view/A4DTYXXpVsNqS1TuUwdb/}{Gantt Roser Prezi}
 \item \href{https://docs.google.com/spreadsheets/d/1TWLecaDlCCRT2Sj8MxZY4WaqSHaYyOZjTOToTy9Fnog/edit#gid=0}{Neurotwin directory.}
\end{itemize}
\begin{subsubsub}
\subsubsection{Deliverables}
\begin{itemize}
    \item \href{https://www.dropbox.com/home/998%20-%20neurotwin.eu%20repository/05%20-%20Deliverables?preview=NEUROTWIN+Deliverable+-+D5.1+Quality+Guide_v1.7_CG.docx}{Deliverable D5.1} (Last Version).
\end{itemize}
\end{subsubsub}

\subsubsection{Galvani}
\begin{itemize}
\item \href{https://www.dropbox.com/home/999%20-%20galvani-lab.eu%20repository/01%20-%20Galvani%20Basic%20Starter%20Kit?preview=Galvani+Tech+Proposal+855109--SEALED-PROPOSAL+short.pdf}{Galvani Grant Agreement TECH (short).}
    \item \href{https://docs.google.com/document/d/18_DfI75pC1e6bMpWSiBY7LVz0cBSdLIWyHygTgJ6p3w/edit}{Galvani Lab. Tasks.}
    \item \href{https://docs.google.com/spreadsheets/d/14srwfkr06nShSOCOmbv2nUXXlONLYjd6/edit?rtpof=true#gid=1839498795}{Galvani Gantt.}
     \item \href{https://docs.google.com/spreadsheets/d/1hmYrOKPDdvS4GNvrOgNENP8p-WfLTDdvGxLtxY_3WpU/edit#gid=0}{Galvani directory.}
\end{itemize}

\subsection{Other documents}
\begin{itemize}
   \item \href{https://docs.google.com/document/d/1TGVl9pppeMySwDjRnmjSmvmIVzkaIl4oBJjY9_VCWMU/edit?usp=sharing}{Rules for joined document revisions.}
   \item \href{https://docs.google.com/document/d/1k3vi7rj7HX7gRnTSQ9_zHMdyylKWJEN0YF3pZPfBDv0/edit}{Don´t panic guide.}
  \item \href{https://docs.google.com/document/d/1pyfM1j0SD_pPCVkUv2lWP6-E2Cya8ucYQ6GsUHYTrDE/edit#heading=h.axrlnzwlxave}{Data Management Plan.} 
  \item \href{https://docs.google.com/spreadsheets/d/10FZQ_QXZLdQijgOvBLNV6e2u1zObHv1MfcH_tUQ954g/edit#gid=1824133226}{Clinical research Overview.}
  \item \href{https://docs.google.com/spreadsheets/d/1-9uVHSc9v8hJlkRH3YDWJ-7bG6dIIWNw1KTHn9ufva4/edit#gid=2120321454}{NE Grants.}
  \item \href{https://docs.google.com/spreadsheets/d/181xnkBHL-Oe6InU_A1kzRPJXk24cG3IVgs6_l0440BA/edit#gid=0}{NE research project list}
\end{itemize}




\printbibliography
\addcontentsline{toc}{chapter}{\hspace*{-2.5cm}Bibliography}
\printindex
\clearpage
\backmatter

\end{document}


%NUMERACI� DE LA P�GINA EXTERIOR EXCEPTE EN LA PRIMERA P�GINA DE CADA CAP�TOL
\usepackage{fancyhdr}
\pagestyle{fancy}
\fancyfoot{}
\fancyfoot[RO]{\thepage}
\fancyfoot[LE]{\thepage}


%MUTIPLES �NDEX
%En el pre�mbul
%\usepackage{multind}
%\makeindex{authors}
